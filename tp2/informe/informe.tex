% !TeX root = informe.tex
\documentclass[a4,11pt]{article}
\usepackage[paper=a4paper, left=3cm, right=3cm, bottom=2.5cm, top=2.5cm]{geometry}
\usepackage[spanish]{babel}
\usepackage[utf8]{inputenc}

\usepackage{caption}
\usepackage{caratula} %Paquete para usar la caratula del DC.(*You don't say?*)
\usepackage{listings}

% Esto es un """hack""" para que listings pueda mostrar carácteres especiales en el código embebido (source: https://en.wikibooks.org/wiki/LaTeX/Source_Code_Listings#Encoding_issue )
\lstset{literate=
  {á}{{\'a}}1 {é}{{\'e}}1 {í}{{\'i}}1 {ó}{{\'o}}1 {ú}{{\'u}}1
  {Á}{{\'A}}1 {É}{{\'E}}1 {Í}{{\'I}}1 {Ó}{{\'O}}1 {Ú}{{\'U}}1
  {à}{{\`a}}1 {è}{{\`e}}1 {ì}{{\`i}}1 {ò}{{\`o}}1 {ù}{{\`u}}1
  {À}{{\`A}}1 {È}{{\'E}}1 {Ì}{{\`I}}1 {Ò}{{\`O}}1 {Ù}{{\`U}}1
  {ä}{{\"a}}1 {ë}{{\"e}}1 {ï}{{\"i}}1 {ö}{{\"o}}1 {ü}{{\"u}}1
  {Ä}{{\"A}}1 {Ë}{{\"E}}1 {Ï}{{\"I}}1 {Ö}{{\"O}}1 {Ü}{{\"U}}1
  {â}{{\^a}}1 {ê}{{\^e}}1 {î}{{\^i}}1 {ô}{{\^o}}1 {û}{{\^u}}1
  {Â}{{\^A}}1 {Ê}{{\^E}}1 {Î}{{\^I}}1 {Ô}{{\^O}}1 {Û}{{\^U}}1
  {œ}{{\oe}}1 {Œ}{{\OE}}1 {æ}{{\ae}}1 {Æ}{{\AE}}1 {ß}{{\ss}}1
  {ç}{{\c c}}1 {Ç}{{\c C}}1 {ø}{{\o}}1 {å}{{\r a}}1 {Å}{{\r A}}1
  {€}{{\EUR}}1 {£}{{\pounds}}1,
}

\usepackage{multicol}
\usepackage{nameref}
\usepackage{natbib}
\usepackage[pdftex]{graphicx}
\usepackage{subfigure} %Paquete para crear subfloats para poner varias imagenes en una linea
\usepackage[svgnames, table, usenames, dvipsnames]{xcolor}
\usepackage{tabularx}

\usepackage{a4wide}
\usepackage{amsfonts}
\usepackage{graphicx}
\usepackage{verbatim}
\parindent = 0 pt
\parskip = 11 pt

\definecolor{mygreen}{rgb}{0,0.6,0}
\definecolor{mygray}{gray}{0.25}
\definecolor{myblue}{rgb}{0.2,0.2,0.6}

%Estilos para el c\'odigo fuente
\lstset{
	basicstyle=\ttfamily\scriptsize,
	breaklines=true,
	captionpos=b,
	commentstyle=\color{mygreen},
	escapechar=@,
	extendedchars=true,
	identifierstyle=\color{myblue},
	language=C++,
	numbers=left,
	numberstyle=\tiny\color{mygray},
	stringstyle=\color{orange},	
	tabsize=3
}

\begin{document}

\titulo{TP 2 - Threads}
\fecha{10/06/2015}
\materia{Sistemas Operativos}
\integrante{Gabriel Gramajo}{564/09}{gramajogm@gmail.com}
\integrante{Paula Jimenez}{655/10}{puly05@gmail.com}

\maketitle

\pagenumbering{arabic}
\parindent 2em %Define indentado de cada parrafo
\parskip 4pt %Define el espacio entre parrafos
\renewcommand{\baselinestretch}{1.5}

\,
\clearpage
\hbox{}
\thispagestyle{empty}
%\newpage

%\tableofcontents

\setcounter{section}{0}

\newpage

\section{Ejercicio 1}
Para implementar los locks nos basamos en la solución sin starvation del problema Readers-writers del libro The little book of semaphores.

A continuación vamos a mostrar el código presentado en el libro y vamos a contar de que manera lo adaptamos para usar las variables de condición que proveen los threads.
\subsection{Inicialización}
En el constructor de la clase, inicializamos los semáforos \emph{mutex}, \emph{turnstile} y una variable \emph{readers} que mantiene la cuenta de cuantos readers están accediendo a la sección crítica en cada momento. El semáforo \emph{roomEmpty} que propone el libro, lo transformamos en una condición \emph{roomEmpty} y un semáforo \emph{mutexRoom}, que van a estar asociados en la variable de condición que despierta a un writer cuando no hay ningún reader leyendo.
\lstinputlisting[name=Readers-writers, numbers=left, frame=single, firstline=0, lastline=4]{readers-writers.cpp}

\subsection{Código Escritores}
\lstinputlisting[name=Readers-writers, numbers=left, frame=single, firstline=7, lastline=13]{readers-writers.cpp}



\subsection{Código Lectores}
\lstinputlisting[name=Readers-writers, numbers=left, frame=single, firstline=17, lastline=33]{readers-writers.cpp}

\section{Ejercicio 2}


\end{document}

